% Copyright (C) 2014-2024 by Thomas Auzinger <thomas@auzinger.name>
\documentclass[draft,final,openany]{vutinfth} % Remove option 'final' to obtain debug information.

\usepackage{geometry}
    
\geometry{
  a4paper,
  top=10mm,       % Abstand vom oberen Seitenrand zum Textblock
  bottom=10mm,    % Abstand vom unteren Seitenrand zum Textblock (hier verringert)
  left=25mm,
  right=25mm,
  includehead,    % falls Kopfzeilen berücksichtigt werden sollen
  includefoot     % falls Fußzeilen berücksichtigt werden sollen
}


% Define convenience functions to use the author name and the thesis title in the PDF document properties.
\newcommand{\authorname}{Marie Curious} % The author name without titles.
\newcommand{\thesistitle}{} % The title of the thesis. The English version should be used, if it exists.

% Create the XMP metadata file for the creation of PDF/A compatible documents.
\begin{filecontents*}[overwrite]{\jobname.xmpdata}
\Author{\authorname}                                    % The author's name in the document properties.
                                  % The document's title in the document properties.
\Language{en-US}                                        % The document's language in the document properties. Select 'en-US', 'en-GB', or 'de-AT'.
\Keywords{a\sep list\sep of\sep keywords}               % The document's keywords in the document properties (separated by '\sep ').
\Publisher{TU Wien}                                     % The document's publisher in the document properties.
\Subject{Thesis}                                        % The document's subject in the document properties.
\end{filecontents*}

% Load packages to allow in- and output of non-ASCII characters.
\usepackage{lmodern}        % Use an extension of the original Computer Modern font to minimize the use of bitmapped letters.
\usepackage[T1]{fontenc}    % Determines font encoding of the output. Font packages have to be included before this line.
\usepackage[utf8]{inputenc} % Determines encoding of the input. All input files have to use UTF8 encoding.

% Extended LaTeX functionality is enables by including packages with \usepackage{...}.
\usepackage{amsmath}    % Extended typesetting of mathematical expression.
\usepackage{amssymb}    % Provides a multitude of mathematical symbols.
\usepackage{mathtools}  % Further extensions of mathematical typesetting.
\usepackage{microtype}  % Small-scale typographic enhancements.
\usepackage[inline]{enumitem} % User control over the layout of lists (itemize, enumerate, description).
\usepackage{multirow}   % Allows table elements to span several rows.
\usepackage{booktabs}   % Improves the typesetting of tables.
\usepackage{subcaption} % Allows the use of subfigures and enables their referencing.
\usepackage[ruled,linesnumbered,algochapter]{algorithm2e} % Enables the writing of pseudo code.
\usepackage{lipsum}
\usepackage[dvipsnames,table]{xcolor} % Allows the definition and use of colors. This package has to be included before tikz.
\usepackage{nag}        % Issues warnings when best practices in writing LaTeX documents are violated.
\usepackage{todonotes}  % Provides tooltip-like todo notes.
\usepackage{morewrites} % Increases the number of external files that can be used.
\usepackage[a-2b,mathxmp]{pdfx}      % Enables PDF/A compliance. Loads the package hyperref and has to be included second to last.
\usepackage[acronym,toc]{glossaries} % Enables the generation of glossaries and lists of acronyms. This package has to be included last.
\usepackage{comment}
%\usepackage[onehalfspacing]{setspace}
\usepackage{sans}
\usepackage[version=4]{mhchem}
\usepackage{chemfig}
\usepackage{siunitx}
%\usepackage[backend=biber,style=chem-acs]{biblatex}
\usepackage{booktabs}
\usepackage{array}
\usepackage{float}
\usepackage{longtable}
\usepackage{array}
\usepackage{tabularx} % im Präambel
\usepackage[english]{babel}
% Set PDF document properties
\hypersetup{
    pdfpagelayout   = TwoPageRight,           % How the document is shown in PDF viewers (optional).
    linkbordercolor = {Melon},                % The color of the borders of boxes around hyperlinks (optional).
}

\setpnumwidth{2.5em}        % Avoid overfull hboxes in the table of contents (see memoir manual).
\setsecnumdepth{subsection} % Enumerate subsections.

\nonzeroparskip             % Create space between paragraphs (optional).
\setlength{\parindent}{0pt} % Remove paragraph indentation (optional).

\makeindex      % Use an optional index.
\makeglossaries % Use an optional glossary.
%\glstocfalse   % Remove the glossaries from the table of contents.

% Set persons with 4 arguments:
%  {title before name}{name}{title after name}{gender}
%  where both titles are optional (i.e. can be given as empty brackets {}).
\setauthor{}{\authorname}{}{female}
\setadvisor{}{Benedict Testtube}{}{male}
%\setsecondadvisor{Projektasse(FWF) Dipl.-Inge}{Klara Wögerbauer}{BSc}{female}
% For bachelor and master theses:
\setfirstassistant{}{Molar Massively}{}{female}
%\setsecondassistant{Pretitle}{Forename Surname}{Posttitle}{male}
%\setthirdassistant{Pretitle}{Forename Surname}{Posttitle}{male}

% For dissertations:
%\setfirstreviewer{Pretitle}{Forename Surname}{Posttitle}{male}
%\setsecondreviewer{Pretitle}{Forename Surname}{Posttitle}{male}

% For dissertations at the PhD School and optionally for dissertations:
%\setsecondadvisor{Pretitle}{Forename Surname}{Posttitle}{male} % Comment to remove.
% Required data.
\setregnumber{01234567}
\setdate{01}{01}{2001} % Set date with 3 arguments: {day}{month}{year}.

%\title{} <- der steht in der vutinfth.cls

%\settitle{\thesistitle}{{Exploring possibilities for 2-butanol production in \textit{A. woodii} by exchange of the \textsc{ADH4} promoter}} % Sets English and German version of the title (both can be English or German). If your title contains commas, enclose it with additional curvy brackets (i.e., {{your title}}) or define it as a macro as done with \thesistitle.
%\setsubtitle{Optional Subtitle of the Thesis}{Optionaler Untertitel der Arbeit} % Sets English and German version of the subtitle (both can be English or German).

% Select the thesis type: bachelor / master / doctor.
% Bachelor:
\setthesis{bachelor}
%
% Master:
%\setthesis{master}
%\setmasterdegree{dipl.} % dipl. / rer.nat. / rer.soc.oec. / master
%
% Doctor:
%\setthesis{doctor}
%\setdoctordegree{rer.soc.oec.}% rer.nat. / techn. / rer.soc.oec.

% For bachelor and master:
\setcurriculum{Technical Chemistry}{technische Chemie} % Sets the English and German name of the curriculum.



% Optional reviewer data:
\setfirstreviewerdata{Affiliation, Country}
\setsecondreviewerdata{Affiliation, Country}


\begin{document}
\raggedbottom

\frontmatter % Switches to roman numbering.
% The structure of the thesis has to conform to the guidelines at
%  https://informatics.tuwien.ac.at/study-services

\addtitlepage{english} % English title page.
%\cleardoublepage
\addstatementpage{}
%\cleardoublepage
%\declaration % Fügt die Erklärung zur Verfassung der Arbeit ein
%\begin{danksagung*}
%\todo{Ihr Text hier.}
%\end{danksagung*}

\begin{acknowledgements*}

\end{acknowledgements*}
%\cleardoublepage

\begin{comment}
\begin{kurzfassung}
\todo{Ihr Text hier.}
\end{kurzfassung}
\end{comment}
\begin{abstract}
\todo{Enter your text here.}



\end{abstract}
\newpage
%\cleardoublepage
% Select the language of the thesis, e.g., english or naustrian.
\selectlanguage{english}

% Add a table of contents (toc).
\tableofcontents % Starred version, i.e., \tableofcontents*, removes the self-entry.

% Switch to arabic numbering and start the enumeration of chapters in the table of content.
\mainmatter

%\chapter{Introduction}
%\todo{Enter your text here.}

\chapter{Introduction}

... Als konzeptionelles Beispiel (Abb. \ref{fig:two_images} (a)) für die Anwendung rheologischer Prinzipien auf komplexe, biologische Systeme dient die humorvolle Untersuchung von \cite{Fardin2014RheologyCats} zum Verhalten von Hauskatzen. Der Autor wendet das Deborah-Zahl-Konzept (Abb. \ref{fig:two_images} (b)) an, um zu argumentieren, dass der scheinbare Aggregatzustand eines Systems stark von der gewählten Beobachtungszeitskala abhängt \cite{Fardin2014RheologyCats}. Diese Arbeit verdeutlicht ...

\begin{figure}[htbp]
    \centering
    \begin{subfigure}{0.48\textwidth}
        \centering        
        \caption{Erstes Bild}
        \includegraphics[width=\linewidth]{graphics/UimvLgxKGsuxxPsZH3XjAYrvvKQzERqCK0OxCG1H.jpg}
    \end{subfigure}
    \hfill
    \begin{subfigure}{0.48\textwidth}
        \centering        
        \caption{Zweites Bild}
        \includegraphics[width=\linewidth]{graphics/MeG9AtWTBUNf4Yur4KEGjTfTw8j7zJjvt18BBsZi.jpg}
    \end{subfigure}
    \caption{Vergleich von zwei verschiedenen Darstellungen}
    \label{fig:two_images}
\end{figure}


\section{Beispiele wissenschaftlicher Notationen}

\subsection{Chemische Formeln mit \texttt{mhchem}}

Die Neutralisationsreaktion zwischen Salzsäure und Natronlauge:
\[
\ce{HCl + NaOH -> NaCl + H2O}
\]

Das Säure-Base-Gleichgewicht von Essigsäure in Wasser:
\[
\ce{CH3COOH + H2O <=> CH3COO- + H3O+}
\]

Die Redoxreaktion bei der Eisenkorrosion:
\[
\ce{4Fe + 3O2 + 6H2O -> 4Fe(OH)3}
\]

\subsection{Strukturformeln mit \texttt{chemfig}}

Die Struktur von Ethan (C\textsubscript{2}H\textsubscript{6}):
\[
\chemfig{H-C(-[2]H)(-[6]H)-C(-[2]H)(-[6]H)-H}
\]

Das Benzolmolekül mit delokalisierten $\pi$-Elektronen:
\[
\chemfig{*6(=-=-=-)}
\]

Die funktionellen Gruppen in Glycin (Aminosäure):
\[
\chemfig{H_2N-CH_2-COOH}
\]

\subsection{Einheiten und Messwerte mit \texttt{siunitx}}

Physikalische Konstanten:
\begin{itemize}
    \item Lichtgeschwindigkeit: \qty{299792458}{\meter\per\second}
    \item Avogadro-Konstante: \qty{6.02214076e23}{\per\mole}
    \item Normbedingungen: \qty{273.15}{\kelvin} bei \qty{101325}{\pascal}
\end{itemize}

Messwerte mit Fehlerangaben:
\begin{itemize}
    \item Temperatur: \qty{23.5 \pm 0.1}{\celsius}
    \item Konzentration: \qty{0.150 \pm 0.003}{\mol\per\liter}
    \item Volumen: \qty{50.00 \pm 0.05}{\milli\liter}
\end{itemize}

Bereichsangaben:
\begin{itemize}
    \item Reaktionstemperatur: \qtyrange{20}{25}{\celsius}
    \item pH-Bereich: \numrange{6.5}{7.5}
\end{itemize}

\subsection{Kombinierte Anwendung im Text}

Die Umsetzung von \qty{2.5}{\gram} (\qty{0.025}{\mol}) Natriumcarbonat (\ce{Na2CO3}) 
mit Salzsäure (\ce{HCl}) bei \qty{25}{\celsius} führt zur Bildung von Natriumchlorid, 
Kohlendioxid und Wasser gemäß:

\[
\ce{Na2CO3 + 2HCl -> 2NaCl + CO2 ^ + H2O}
\]

Die freigesetzte Gasmenge beträgt unter Normbedingungen \qty{0.56}{\liter} 
(\qty{0.025}{\mol} \ce{CO2} × \qty{22.4}{\liter\per\mol}).

\chapter{Materials and methods}

\lipsum[1] % Beispieltext


\chapter{Results and Evaluation}


\section{Meine lange Tabelle}

\lipsum[1] % Beispieltext

\begin{longtable}{|p{3cm}|p{3cm}|p{6cm}|}
\caption{Eine lange Tabelle, die über zwei Seiten geht} \\
\hline
\textbf{Spalte 1} & \textbf{Spalte 2} & \textbf{Spalte 3} \\
\hline
\endfirsthead

\multicolumn{3}{c}{{\tablename\ \thetable{} -- Fortsetzung von vorheriger Seite}} \\
\hline
\textbf{Spalte 1} & \textbf{Spalte 2} & \textbf{Spalte 3} \\
\hline
\endhead

\hline
\multicolumn{3}{r}{{Fortsetzung auf nächster Seite}} \\
\endfoot

\hline
\endlastfoot

% Hier beginnt der Tabelleninhalt
Element 1 & Daten A & Beschreibung 1 \\
\hline
Element 2 & Daten B & \lipsum[2][1-2] \\
\hline
Element 3 & Daten C & Beschreibung 3 \\
\hline
Element 4 & Daten D & Beschreibung 4 \\
\hline
Element 5 & Daten E & \lipsum[3][1-2] \\
\hline
Element 6 & Daten F & Beschreibung 6 \\
\hline
Element 7 & Daten G & Beschreibung 7 \\
\hline
Element 8 & Daten H & \lipsum[4][1-2] \\
\hline
Element 9 & Daten I & Beschreibung 9 \\
\hline
Element 10 & Daten J & Beschreibung 10 \\
\hline
Element 11 & Daten K & \lipsum[5][1-2] \\
\hline
Element 12 & Daten L & Beschreibung 12 \\
\hline
Element 13 & Daten M & Beschreibung 13 \\
\hline
Element 14 & Daten N & \lipsum[6][1-2] \\
\hline
Element 15 & Daten O & Beschreibung 15 \\
\hline
Element 16 & Daten P & Beschreibung 16 \\
\hline
Element 17 & Daten Q & \lipsum[7][1-2] \\
\hline
Element 18 & Daten R & Beschreibung 18 \\
\hline
Element 19 & Daten S & Beschreibung 19 \\
\hline
Element 20 & Daten T & \lipsum[8][1-2] \\
\hline
Element 21 & Daten U & Beschreibung 21 \\
\hline
Element 22 & Daten V & Beschreibung 22 \\
\hline
Element 23 & Daten W & \lipsum[9][1-2] \\
\hline
Element 24 & Daten X & Beschreibung 24 \\
\hline
Element 25 & Daten Y & Beschreibung 25 \\
\hline
Element 26 & Daten Z & \lipsum[10][1-2] \\
\hline
\end{longtable}

\lipsum[11-15] % Weitere Beispieltexte nach der Tabelle


\chapter{Discussion}

\lipsum[11-15]

\chapter{Outlook}

% Remove following line for the final thesis.
%\input{intro.tex} % A short introduction to LaTeX.

\lipsum[11-15]

\backmatter

% Declare the use of AI tools as mentioned in the statement of originality.
% Use either the English aitools or the German kitools.
%\begin{aitools}
%\end{aitools}

%\begin{kitools}
%\todo{Enter your text here.}
%\end{kitools}

% Use an optional list of figures.
\listoffigures % Starred version, i.e., \listoffigures*, removes the toc entry.

% Use an optional list of tables.
%\cleardoublepage % Start list of tables on the next empty right hand page.
\listoftables % Starred version, i.e., \listoftables*, removes the toc entry.

% Use an optional list of alogrithms.
%\listofalgorithms
%\addcontentsline{toc}{chapter}{List of Algorithms}

% Add an index.
%\printindex

% Add a glossary.
%\printglossaries

% Add a bibliography.
\bibliographystyle{apalike}
\bibliography{literatur}


\appendix
\renewcommand{\thechapter}{\Alph{chapter}}
\renewcommand{\chaptername}{Appendix}

%\chapter*{Appendix A: HPLC Standards}
%\renewcommand{\thechapter}{A}
%\addcontentsline{toc}{chapter}{Appendix A: HPLC Standards}
%\label{app:standards}

\chapter*{Appendix A: HPLC Standards}
\addcontentsline{toc}{chapter}{Appendix A: HPLC Standards}
\renewcommand{\thechapter}{\Alph{chapter}}
\label{app:standards}


\chapter*{Appendix B: Primer}
\renewcommand{\thechapter}{\Alph{chapter}}
\addcontentsline{toc}{chapter}{Appendix B: Primer}

\label{app:primer}



\chapter*{Appendix C: Tools Used}
\renewcommand{\thechapter}{\Alph{chapter}}
\addcontentsline{toc}{chapter}{Appendix C: Tools Used}
    

%\chapter*{Appendix C: Agarose Gel Electrophoresis of PCR Products}
%\addcontentsline{toc}{chapter}{Appendix C: Agarose Gel Electrophoresis of PCR Products}



\end{document}